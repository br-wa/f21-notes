\section*{Lecture 3: Work and Heat (Voigt)}
\setcounter{section}{3}
\setcounter{defn}{0}
\setcounter{subsection}{0}
\setcounter{defncontainer}{0}

Recall from last time that for a closed system, $\Delta U = q + w$, i.e. heat and work are the only ways to transfer energy across the boundary.
This relation is quite subtle; while $U$ is a state function, $q + w$ are \emph{path-dependent}. In particular, $\oint q\,\text dC$ is nonzero along some $P-V$ loop.

\begin{defn}
	A system is at \vocab{equilibrium} if certain \vocab{state functions} are uniform throughout the system. State functions being constant are \vocab{indicators} of equilibrium.
\end{defn}

Note that this is not the exact same as the ``state functions do not change over time," but the lack of a gradient usually indicates equilibrium.
Note also that state functions are not well-defined unless the system is at equilibrium.

\begin{exm}
	The following are useful state functions:
	\begin{itemize}
		\item Pressure $P$,
		\item Volume $V$,
		\item Temperature $T$, 
		\item Internal energy $U$,
		\item Chemical potential $\mu = \frac{G}{n}$.
	\end{itemize}
\end{exm}

\begin{defn}*
	A state function is \vocab{intrinsic} if it is a density function, while a state function is \vocab{extrinsic} if it is defined by some sum/integration over the system.
\end{defn}

Now, we can ask the following:

\begin{que}*
	What is the minimum number of state functions needed to fully describe a system?
\end{que}

It turns out the answer is known:

\begin{thm}
	If there are $N_s$ species in a system, $N_s+2$ (non-redundant) state functions uniquely determine the state.
\end{thm}

\begin{defn}*
	An \vocab{equation of state} is an equation between some state functions, where all state functions not in the equation are assumed to be held constant.
\end{defn}

\begin{exm}*
	[Ideal Gas Law]
	$PV \sim nT$ is an equation of state. We may also write $V\sim P^{-1}$, $V\sim n$, $V\sim T$. In particular, for a closed system (in which $n$ is thus constant), the Gas Law defines a \emph{surface} in the 3d space $(P,V,T)$.
\end{exm}

\begin{exm}*
	[Van der Waals]
	We can also write $RT = \parens*{P + \frac a{\ol V^2}}(\ol V - b)$. This is still the same number of state variables; however, the equation is now more complex. 
\end{exm}

In general, state equations can get arbitrarily complex. However, since $N_s+2$ state functions are enough, we cannot create state functions with arbitrarily many variables.

\subsection{Changing State Functions}

In the ideal gas case, recall that a closed system is uniquely determined by its $P,V$. Thus, we can represent states as points on a graph with axes $(V,P)$. Note that points are only well-defined if they are at equilibrium.

\begin{defn}
	A transition is \vocab{reversible} if it can be continuously undone. A transition is \vocab{irreversible} otherwise.
\end{defn}

Irreversible systems are often at non-equilibrium, which makes them harder to study. We will mostly focus on reversible situations here.
