\section*{Lecture 1: Introduction (Voigt)}
\setcounter{section}{1}

We have a ``one mask policy" where at most one person can take off their mask at once. So you can take your mask off to ask questions, but then you should let Professor Voigt put on his mask first. For contact tracing purposes you should remember which quadrant of the classroom you sit in and try to keep this constant over time. TA Kwan is recording, but not simul-casting, the lectures.

Professor Voigt is teaching the first half of the class, and will have Office Hours on Friday by appointment. His office is in 500 Tech Square. Professor Birnbaum's office is in the Koch Institute. The first half will be on classical thermodynamics, while the latter half is based on statistical mechanics. The first half does not have a textbook -- Eisenberg (???)'s is quite good if we want one. There are lecture notes, which contain mostly figures and equations. For the second half, Dill and Bromberg have a good graduate-level book.

There are four exams. The first and third are take-home and worth fewer points, while the midterm is 7-9 in a large room in Building 10 (probably 10-250), and the final is also in-person. A small amount of cheat sheets are allowed for the in-person exams, while the take-home ones are open-everything.

\subsection{Introduction to BE}

It is useful to note that Bioengineering and Biomedical Engineering are not the same field. Bioengineering is considerably newer and has a myriad of applications.

We are interested in a lot of problems, such as ligand binding and so forth, that require an understanding of thermodynamics. However, because we are interested in doing things at a molecular level, we need to understand statistical mechanics as well.


