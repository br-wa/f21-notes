\section*{Lecture 12: Gibbs Free Energy (Voigt)}
\setcounter{section}{12}
\resetcounter{subsection}
\resetcounter{defn}
\resetcounter{defncontainer}

Recall from last time that we claimed the following:

\begin{law}
	[Second Law, Formalized]
	$\Delta S_{\textnormal{univ}} = \Delta S_{\textnormal{sys}} + \Delta S_{\textnormal{surr}} \geq 0$ for all spontaneous processes, with equality for reversible processes.
\end{law}

\subsection{Free Energy}

It is in general hard to compute $\Delta S_{\text{surr}}$. Suppose our process occurs at \textbf{constant temperature and constant pressure}. Then, \[
	\Delta S_{\text{surr}} = \frac{q_{\text{surr}}}{T} = -\frac{q_\text{sys}}{T} = -\frac{\Delta H_{\text{sys}}}{T}.
\]
(The first equality needs constant temperature, while the last one needs constant pressure).
It follows that \[
	\Delta S_{\text{univ}} \geq 0 \iff \Delta S_{\text{sys}} - \frac{\Delta H_{\text{sys}}}{T} \geq 0\iff \Delta H_{\text{sys}} - T\Delta S_{\text{sys}} \leq 0.
\]
We thus define:
\begin{defn}
	Define the \vocab{(Gibbs) free energy} $G$ of a system as $G = H - TS$.
\end{defn}

Then, 

\begin{thm}
	An isothermal, isobaric process is spontaneous if and only if $\Delta G \leq 0$.
\end{thm}
