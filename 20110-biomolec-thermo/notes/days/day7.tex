\section*{Lecture 7: Reactions (Voigt)}
\setcounter{section}{7}
\setcounter{subsection}{0}
\setcounter{defn}{0}
\setcounter{defncontainer}{0}

\begin{defn}
	Let $\Delta_f H^\circ$ for some compound denote the \vocab{standard enthalpy of formation}, i.e. the change in enthalpy when the compound is formed from its constitutent atoms in their ``most stable forms," at standard conditions (i.e. 298 K and 1 atm).
\end{defn}

\begin{fact}
	For a reaction, the \vocab{heat of reaction} is: \[
		\Delta_r H^\circ = \sum_{\textnormal{products}} v\Delta_f H^\circ - \sum_{\textnormal{reactants}} v\Delta_f H^\circ,
	\]
	where $v$ is the coefficient of some reactant or product in the reaction.
\end{fact}

\begin{defn}*
	For a reaction and a fixed pressure $P$, 
	\[
		\Delta C_P = \frac{\partial \Delta H}{\partial T}|_P = \sum_{\textnormal{products}} vC_P - \sum_{\textnormal{reactants}} vC_P.
	\]
\end{defn}

\begin{law}*
	[Kirchoff]
	Suppose a reaction happens at the same pressure and two temperatures $T_1, T_2$, with reaction enthalpies $\Delta_r H_1$ and $\Delta_r H_2$, respectively. Then, \[
		\Delta_r H_2 = \Delta_r H_1 + \Delta C_P (T_2-T_1).
	\]
\end{law}
