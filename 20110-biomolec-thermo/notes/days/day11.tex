\section*{Lecture 11: The Second Law of Thermodynamics (Voigt)}
\setcounter{section}{11}
\resetcounter{subsection}
\resetcounter{defn}
\resetcounter{defncontainer}

Recall that from last time we derived $\frac{\Delta S}{n} = R\ln \frac{V_2}{V_1}$ for an isothermal expansion of an ideal gas.
However, we must note that $\Delta S_{\text{sys}}$ does not uniquely determine whether something will happen. 
For example, consider protein folding.

Recall that $\d S_{\text{surr}} = \frac{\d q_{\text{surr}}}{T}$.
In particular, when the reaction is reversible, $\d q_{\text{surr}} = - \d q_{\text{sys}}$, so \[
	\d S_{\text{surr}} = -\frac{\d q_{\text{sys}}}{T} = - \d S_{\text{sys}}.
\]
So, 
\begin{fact}
	$\Delta S_{\text{univ}}$ for all reversible processes.
\end{fact}

On the other hand, suppose we have an isothermal expansion, but surroundings are a vacuum. Then, the process is irreversible, since it there is no external force to reverse it. Now, $\Delta S_{\text{surr}} = 0$, so $\Delta S_{\text{univ}} = R \ln \frac{V_2}{V_1} > 0$.

\begin{law}
	[Second Law]
	The entropy of the universe increases as a result of any spontaneous process.
\end{law}	

\begin{defn}
	A \vocab{spontaneous} process is one that occurs without external work.
\end{defn}

Thus we see that $\Delta S_{\text{univ}}$ is a \vocab{condition of spontaneity}.

\begin{defn}
	For a reaction, define its \vocab{reaction entropy} as $\Delta_r S = \sum\limits_{\text{products}} v S - \sum\limits_{\text{reactants}} v S$.
\end{defn}

\begin{fact}*
	Not all $S = 0$ at $T = 0$. At $T = 0$ the substance forms a crystal, whose structure can be in one of multiple microstates. 
\end{fact}

\begin{defn}
	The \vocab{residual entropy} of a substance is its entropy at $T = 0$.
\end{defn}

\begin{exm}
	[Residual Entropy of Carbon Monoxide]
	Each molecule can be in either C=O or O=C form, so \[
		S = k_B \ln W = k_b \ln 2^{N} = Nk_B \ln 2 = nR \ln 2.
	\]
\end{exm}


