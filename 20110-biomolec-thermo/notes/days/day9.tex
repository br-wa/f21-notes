\section*{Lecture 9: Spontaneity (Voigt)}
\setcounter{section}{9}
\setcounter{subsection}{0}
\setcounter{defn}{0}
\setcounter{defncontainer}{0}

\begin{defn}
	We say a process $A\to B$ is \vocab{spontaneous} if it goes in the forward direction without external work.
\end{defn}

Note that this is independent of the sign of $\Delta H$. For example, glucose combustion (exothermic) is spontaneous, while cold packs (endothermic) are also spontaneous.
The following is a real experiment done at the UCSF pharmacy school:

\begin{exper}*
	Take two blocks of metal next to each other at the same temperature.
	They will not spontaneously become different temperatures.
\end{exper}

We can do a more interesting version of this:

\begin{exper}*
	Take two blocks of metal next to each other at different temperatures.
	They will spontaneously become the same temperature.
\end{exper}

\begin{que}
	Why does equilibrium happen to have the same temperature throughout?
	Similarly, why does a gas expand at equilibrium?
\end{que}

\begin{fact}*
	Suppose one has a few (say, 5) gas particles in a box. There is a nonzero probability that all the particles clump into one corner, but the probability is much lower than the probability of them being all spread out.
\end{fact}

\begin{defn}
	Energy and mass tend to \vocab{dissipate}, or spread out, at equilibrium.
\end{defn}


