\section*{Lecture 9: Human Mendelian Traits I (Reddien)}
\setcounter{section}{9}
\setcounter{subsection}{0}
\setcounter{defn}{0}
\setcounter{defncontainer}{0}

\subsection{Centromere Linkage}

Suppose we have two unlinked genes A, B (say on separate chromosomes).
Then, note that if B is far from the chromosome, and we have a Bb, we have the possibility that there is a crossing-over event between the centromere and, say gene B.

This will produce tetratypes. In particular:
\begin{fact}
	If the proportion of tetratypes is at least 2/3, then one of the genes is unlinked to the tetratype. Otherwise, \emph{if one gene is closely linked to its centromere}, then the other gene's distance to its centromere is \[
		100 \cdot \frac{\textnormal{TT}}{2\cdot \textnormal{total}}.
	\]
\end{fact}

\subsection{Pedigree Analysis}

\begin{fact}*
	In general, there are a few assumptions we can make:
	\begin{itemize}
		\item (Complete penetrance) Individuals express phenotypes as predicted by their genotype.
		\item For traits that are rare, we assume individuals who marry into the pedigree (and are not founders) are not carriers.
		\item There are no new mutations in the pedigree.
		\item (Correct paternity) The pedigree is correct.
	\end{itemize}
	From here, we just test hypotheses (autosomal dominant, autosomal recessive, x-linked dominant, x-linked recessive) for consistency.
	We may also need to test the rarity assumption.
\end{fact}

\subsection{Conditional Probability}

Now, suppose we want to know what the probability of someone being a carrier is.
If we know their parents, we can compute the probability \emph{a priori}.
Otherwise, if we have other information (either from siblings or children), we can use Bayes's Theorem to compute the conditional probability of someone being a carrier.
