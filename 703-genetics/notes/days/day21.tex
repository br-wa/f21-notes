\section*{Lecture 21: Gene Regulation (Gehring)}
\setcounter{section}{21}
\resetcounter{subsection}
\resetcounter{defn}
\resetcounter{defncontainer}

Suppose we are now interested in how regulation works. Note that we can split the genome into loci that are similarly regulated.

\begin{defn}
	We can insert \vocab{reporter genes}, usually coding for fluorescent proteins, into loci to see if they are on or off.
\end{defn}

Another strategy we can do is delete specific regulatory regions and see what happens to expression.

\begin{defn}
	A regulatory locus can be the following:
	\begin{itemize}
		\item An \vocab{upstream activating sequence (UAS)},
		\item An \vocab{upstream repressor sequence (URS)}, or
		\item A \vocab{TATA box}, where RNA Polymerase II binds to start transcription.
	\end{itemize}
\end{defn}

Going back to the galactose pathway from earlier, we have: \[
	\text{galactose} \dashv \opera{Gal80} \dashv \opera{Gal4} \to \opera{Gal1}.
\]
Furthermore, we can identify the Gal1 locus as having regulators as UAS, a URS, a TATA box, and then Gal1 in that order.
We have the following behavior:
\begin{itemize}
	\item Gal4 always binds to the UAS.
	\item In the absence of galactose, Gal80 binds to Gal4, preventing it from activating behavior.
	\item In the presence of galactose, Gal3 binds to Gal80, preventing it from binding to Gal4 and thus allowing Gal4 to induce the UAS.
	\item In the presence of glucose, Mig1 binds to the URS and suppresses transcription.
\end{itemize}

\begin{defn}
	We say that a protein that regulates the transcription of another gene is a \vocab{transcription factor (TF)}. 
	Transcription factors usually have a \vocab{binding domain (BD)} that allows proteins to regulate the TF, and a \vocab{activation domain (AD)} that binds to the DNA.
\end{defn}

\begin{exper}
	In \emph{Drosophila} there is a gene called \emph{sonic hedgehog (SHH)} that is involved in limb development and expressed in limbs. It is activated by an enhancer called \emph{ZRS} that is present in all vertebrates (including snakes).
	Amazingly, ZRS is 850 kb upstream of SHH. So why is it there?

	We can perform the following experiment: Add some synthetic DNA consisting of a species-specific ZRS and a reporter gene (e.g. lacZ), that way we can tell when ZRS performs its enhancing job.
	We get the following results:
	\begin{itemize}
		\item In mice, lacZ is expressed in limbs.
		\item In finned vertebrates (e.g. fish), lacZ is also expressed in limbs.
		\item In a bunch of other limbed vertebrates, lacZ is also expressed in limbs.
		\item In snakes, lacZ is not consistently expressed.
	\end{itemize}
	We then perform the following: replace mouse ZRS with snake ZRS. 
	Then, we get \emph{serpentized} mice, i.e. those without limbs.
	In fact, when we compare snake ZRS with other ZRS, we notice a 17 BP deletion, which occurs when a TF called ETSI binds.
	This prevents the binding of TFs, thus preventing enhancing behavior.
\end{exper}
