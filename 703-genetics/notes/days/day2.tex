\section*{Lecture 2: Independent Assortment (Reddien)}
\setcounter{section}{2}
\setcounter{defn}{0}
\setcounter{defncontainer}{0}
\setcounter{subsection}{0}

\subsection{Equal Segregation}

We return to our study of the ``paralyzed" gene from before, and we assume flies are diploid:

\begin{exper}*
	We cross true-breeding paralyzed and wild-type flies, and the F1 generation is completely non-paralyzed. When we cross an F1 fly with another F1 fly, around a quarter of the flies end up being paralyzed.
\end{exper}

We can now analyze the genotypes. Suppose we hypothesize that the F0 cross was para\textsuperscript{ts}/para\textsuperscript{ts} ($\x$) para\textsuperscript{wt}/para\textsuperscript{wt}, which means the entire F1 generation was para\textsuperscript{ts}/\textsuperscript{wt}. We then hypothesize that whenever the genotype contains both para\textsuperscript{ts} and para\textsuperscript{wt} copies, the wild-type phenotype is observed.

\begin{defn}
	In a diploid, the genotype of a specific gene is \vocab{heterozygous} if distinct alleles are present, while such a genotype is \vocab{homozygous} of two identical alleles are present. A specimen that is heterozygous for a specific gene is a \vocab{heterozygote} (resp. \vocab{homozygote}). A phenotype is \vocab{dominant} if it is observed in heterozygotes containing at least one copy of the corresponding allele. On the other hand, a phenotype is \vocab{recessive} if it is only observed in homozygotes containing two copies of the corresponding allele.
\end{defn}

How does recessivity work? Recall that the para gene actually encodes a sodium channel. So, it is possible that in heterozygotes, the wild-type allele encodes for enough functional sodium pumps that movement can occur as normal. In fact, in many genes, one functional allele is sufficient.

To analyze the F2 we need the following:

\begin{law}
	[Equal Segregation/Mendel's First Law]
	In diploid reproduction, allele pairs separate during gamete formation and randomly unite (one copy from each parent) during fertilization.
\end{law}

Now we consider the F1 cross of para\textsuperscript{wt}/para\textsuperscript{ts} ($\x$) para\textsuperscript{wt}/para\textsuperscript{ts}. Our resulting possibilities are:
\begin{itemize}
	\item para\textsuperscript{wt}/para\textsuperscript{wt},
	\item para\textsuperscript{wt}/para\textsuperscript{ts},
	\item para\textsuperscript{ts}/para\textsuperscript{wt},
	\item para\textsuperscript{ts}/para\textsuperscript{ts}.
\end{itemize}
Furthermore, Equal Segregation tells us that each of these results are equally possible. Note that the middle two classes are the same genotype. So, in fact the heterozygous genotype has a 50\% chance of occuring.

Now, if we look at phenotypes, we note that only the last result leads to a paralyzed fly, so only one in four flies is paralyzed, which is as observed.
Thus, our hypothesis that paralysis is a recessive trait governed by a signel para gene.

\subsection{Independent Assortment}

Suppose we now consider the following two hypotheses:

\begin{itemize}
	\item Paralysis is a recessive trait governed by two genes, para and shibere, and to be paralyzed, a fly needs to be homozygous in both.
	\item Paralysis is a recessive trait governed by two genes, para and shibere, and to be paralyzed, a fly needs to be homozygous in at least one. 
\end{itemize}


Then, our F0 cross would be:
\begin{center}
	para\tss{ts}/para\tss{ts};shib\tss{ts}/shib\tss{ts} ($\x$) para\tss{wt}/para\tss{wt};shib\tss{wt}/shib\tss{wt}.
\end{center}
In that case, the F1s would be heterozygous in both, and thus must be para\tss{wt}/para\tss{ts};shib\tss{wt}/shib\tss{ts}. To analyze an F1 ($\x$) F1 cross, we need the following:

\begin{law}
	[Independent Assortment/Mendel's Second Law]
	For \vocab{unlinked} genes, inheritance of a given allele of one gene and inheritance of a given allele of the other gene are independent events.
\end{law}

Now, the possible F1 gametes are:

\begin{itemize}
	\item para\tss{wt}/shib\tss{wt},
	\item para\tss{wt}/shib\tss{ts},
	\item para\tss{ts}/shib\tss{wt}, 
	\item para\tss{ts}/shib\tss{ts}.
\end{itemize}

We can then write out all $16$ possibilities and actually count things, or we can use expected value. Note that $P(\text{homozygous para\tss{ts}}) = 0.25$ and $P(\text{homozygous shib\tss{ts}}) = 0.25$, and furthermore these events are independent. So, the probability of both is $0.25^2 = 1/16$, while the probability of either is $0.25 + 0l25 - 1/16 = 7/16$. It follows that our two para + shibere hypotheses expect $\frac 1{16}$ and $\frac 7{16}$ of the F2 flies to be paralyzed, respectively.

\subsection{Hypothesis Testing}

Suppose now that our actual data is $4$ paralyzed, $12$ non-paralyzed. Then, we can test the last hypothesis using a chi-squared test.

\begin{defn}
	[Chi-Squared]
	If we have events $A_1, \ldots, A_k$ with expected frequences $f_1,\ldots, f_k$ and observed frequencies $o_1, \ldots, o_k$, then the \vocab{chi-squared statistic} $\chi^2$ is \[
		\chi^2 = \sum \frac{(o_i-f_i)^2}{f_i}.
	\]
	In a chi-squared test, the \vocab{degrees of freedom (df)} is defined as the number of independent normally distributed variables (usually $k-1$). We can then determine the $p$-value of this \[
		\mathbb P(\text{chi-squared} \geq \chi^2 \mid \text{df degrees of freedom hypothesis is true}).
	\]
\end{defn}

In the case of our last hypothesis, we get \[
	\chi^2 = \frac{(4-7)^2}{7} + \frac{(12-9)^2}{9} = 2.28.
\]
The degrees of freedom is $k-1 = 1$, so we can look up in a table and find $p > 0.05$, so we cannot reject the last hypothesis and need more data to differentiate. In general, we can use a \vocab{power analysis} to determine the amount of data needed to differentiate between two hypotheseso.

