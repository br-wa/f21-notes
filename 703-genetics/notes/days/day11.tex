\section*{Lecture 11: Human Mendelian Traits III (Reddien)}
\setcounter{section}{11}
\resetcounter{subsection}
\resetcounter{defn}
\resetcounter{defncontainer}

\subsection{Linkage in Pedigrees}

Suppose we have a gene X (with D denoting allele corresponding to the dominant phenotype and + the wild-type).
Say we also have some SSR with alleles A, B, C, E, F, and so forth (we skip D because it is already used).

We proceed as follows:

\begin{enumerate}
	\item First, determine mode of inheritance (we assume autosomal dominance for this example).
	\item Ascribe genotypes (we can do standard pedigree analysis for gene X, and then use the molecular tests from last time).
	\item Identify informative meioses (to be defined later).
	\item Identify recombinant and non-recombinant gametes.
	\item Compute a LOD score.
\end{enumerate}

\begin{exm}*
	[Canine Narcolepsy]
	There is a pedigree shown in class that I will not show here. 
	We identify the \emph{canarc} gene.
\end{exm}

\subsection*{Informative Meioses}

Ordinarily, we'd like to do a test cross of the form AB/ab (\x) ab/ab, which lets us observe the genotype of all gametes.
Note that an Ab/aB (\x) ab/ab also works, for example.

\begin{defn}
	An \vocab{informative meiosis} is one in which there are two alleles of the trait gene and 2 alleles of the marker present, and it is possible to determine which alleles were transmitted in the meiosis.
\end{defn}

\begin{exm}*
	Suppose X is autosomal dominant for some rare trait, with a father having genotype D/+ and with SSR1 genotype A/B.
	Meanwhile, the mother is +/+ (as the trait is rare) and has SSR1 genotype C/E.
	Their daughter is D/+ A/E, which is an informtive meiosis since the daughter must have received DA from the father and +E from the mother.
\end{exm}

\begin{exm}*
	Suppose X is autosomal dominant and rare still, with the father being D/+ A/A, and the mother being +/+ B/B. The daughter is D/+ A/B, which is not informative because we do not know which A was received.
\end{exm}

\begin{exm}*
	Same X, with the father D/+ A/B, and the mother +/+ A/B. The daughter D/+ A/B, which is not informative because we could have gotten DA or DB from the father.
\end{exm}

\begin{exm}*
	Same X, with the father D/+ A/B, the mother +/+ A/B, and the daughter D/+ A/A. This is informative because we must have gotten DA from the father.
\end{exm}	

\begin{exm}*
	Same X, with the father D/+ A/B, the mother +/+ A/B, and the daughter +/+ B/B. This is informative because we must have gotten +B from the father.
\end{exm}

Note that this is not enough. If the individual was A\sim B/a\sim b, then A\sim b would be recombinant. On the other hand, if the individual was A\sim b/a\sim B, then A\sim b would not be recombinant.

\begin{defn}
	The \vocab{phase} denotes the alleles of an individual that were inherited from the same gamete.
\end{defn}

Sometimes we can try to identify an individual's phase by looking at their parents' genotypes.

\begin{exm}*
	Suppose we have $1$ recombinant and $3$ non-recombinant. 
	We hypothesize a linkage at $\Theta = 25$ centiMorgans.
\end{exm}

