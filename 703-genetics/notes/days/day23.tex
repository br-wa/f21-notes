\section*{Lecture 23: Editing the Genome I: Recombinant DNA (Gehring)}
\setcounter{section}{23}
\resetcounter{subsection}
\resetcounter{defn}
\resetcounter{defncontainer}

\begin{defn}
	\vocab{Recombinant DNA} is the combination of DNA from different sources.
\end{defn}

\begin{fact}
	[History of Recombinant DNA]
	\begin{itemize}
		\item[1972] First recombinant DNA molecules made (viral gene cloned into bacteria).
		\item[1974] Scientists call for moratorium of recombinant DNA research, ask NIH for guidelines.
		\item[1975] Asilomar conference on safe use of recombinant DNA; NIH guidelines issued.
		\item[1976] Cambridge institutes moratorium on recombinant DNA experiments.
		\item[1977] Cambridge recombinant DNA moratorium lifted, Cambridge Recombinant DNA Technology Ordinance passed.
		\item[1982] Humulin (recombinant insulin) comes to market.
		\item[1990] Rennin (first recombinant food ingredient) comes to market.
	\end{itemize}
\end{fact}

\begin{defn}
	[Recombinant RNA Toolbox]
	Recombinant RNA uses the following:
	\begin{itemize}
		\item A \vocab{ligase} joins strands of DNA.
		\item A \vocab{DNA polymerase} copies DNA.
		\item A \vocab{reverse transcriptase} copies RNA into DNA.
		\item A \vocab{restriction endonuclease} cuts DNA at a specific 4-8 base pair \vocab{recognition site}.
		\item An \vocab{exonuclease} chews DNA from one end.
	\end{itemize}
\end{defn}

\begin{defn}
	We will often make recombinant DNA in circular plasmid, which usually starts off with the following:
	\begin{itemize}
		\item An \vocab{origin of replication (ori)}.
		\item A \vocab{cloning site}, where DNA can be added.
		\item A \vocab{selectable marker}, usually ampR (ampicillin resistance) which can allow for screens for bacteria with the plasmid.
	\end{itemize}
	We will add this to \vocab{competent} bacteria, which can take up plasmids.
\end{defn}

\begin{exper}
	[Recombinant DNA]
	The following procedure transforms bacteria with plasmids containing a desired gene:
	\begin{enumerate}
		\item Cut the plasmid and add the new DNA, and use a ligase to fuse everything together.
		\item Transform competent bacteria via electric pulse, heat shock, or transduction.
		\item Screen for bacteria that have taken up the plasmid (e.g. grow on ampR).
	\end{enumerate}
\end{exper}

\begin{exper}
	Suppose one has the \emph{rennet} gene from cows, and we want to express it in a bacteria.
	One thing we can do is just buy some synthetic DNA.
	But we can also do the following:
	\begin{enumerate}
		\item Isolate cow RNA (from a tissue with high rennet expression) and use a TTTT primer and reverse transcriptase to make a cDNA library of all present RNA.
		\item Use PCR to amplify only the rennet DNA.
		\item Remove the primers and replace them with the same primers with EcoRI tags, so that the rennet DNA now has EcoRI tags on the ends.
		\item Use EcoRI to cut the plasmid, and fuse. 
	\end{enumerate}
\end{exper}

We can speed this up a bit:

\begin{exper}
	[Gibson Cloning]
	In \vocab{Gibson cloning}, one starts off with strands $S_1,\ldots, S_k$ such that $(S_i, S_{i+1})$ have \emph{sequence homology}, i.e. the ends of $S_i$ and $S_{i+1}$ match up. Then, we can use some exonuclease to chew off the 5' ends of every strand, which then allows for annealing.
\end{exper}

\begin{exper}
	[Golden Gate Cloning]
	In \vocab{Golden Gate cloning}, one uses a \vocab{Type II restriction enzyme}, which cuts after the recognition site (as opposed to during the restriction site), which creates controllable overhangs. Thus, by manipulating the overhangs one can combine a bunch of genes in order.  
\end{exper}
