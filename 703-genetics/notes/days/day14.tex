\section*{Lecture 14: Mutations II (Gehring)}
\setcounter{section}{14}
\resetcounter{subsection}
\resetcounter{defn}
\resetcounter{defncontainer}

\begin{defn}*
	A mutation in which a purine changes into another purine (A \leftrightarrow G) or a pyrimidine changes into another pyrimidine (C \leftrightarrow T) is called a \vocab{transition}. A mutation in which a purine changes into a pyrimidine (or vice versa) is a \vocab{transversion}.
\end{defn}

\subsection{Inversions}

Last time we discussed inversions, in which parts of the chromosome rearrange. 

\begin{defn}*
	In a \vocab{paracentric} inversion, the centromere is outside the inversion. On the other hand, in a \vocab{pericentric} inversion, the centromere is inside the inversion. Furthermore, an individual is called a \vocab{inversion heterozygote} if they have a normal chromosome and an inversion chromosome.
\end{defn}

\begin{exm}
	Suppose we have an inversion heterozygote where an individula has a paracentric inversion.
	This forms an \vocab{inversion loop}, which suppresses crossing over in nearby genes. However, crossing over can still occur in the inversion loop.
	As an example, suppose the normal is (Centromere)ABCDEF and the inversion is (Centromere)ABDCEF. Then, if we have a crossing over between C and D, we get the two following fragments:
	\begin{itemize}
		\item An \vocab{acentric fragment} FEDCEF.
		\item A \vocab{dicentric bridge} FEDCBA(Centromere)ABCDBA(Centromere)ABDCEF.
	\end{itemize}
	The acentric fragment will not continue in meiosis, since the lack of centromere makes it not attach to the mitotic spindle.
	On the other hand, the dicentric fragment has two centromeres, so it gets pulled in both directions by spindles.
	Thus the bridge (ABCDBA) breaks. Suppose it becomes ABCD/BA. Then, after meiosis, we get the four gametes:
	\begin{itemize}
		\item (Centromere)ABCDEF
		\item (Centromere)ABCD
		\item (Centromere)AB
		\item (Centromere)ABDCEF
	\end{itemize}
	The two parental gametes (ABCDEF and ABDCEF) work fine, but the recombinant gamates become either inviable or create inviable zygotes.
\end{exm}

\begin{fact}
	In an inversion heterozygote, only parental gametes appear, i.e. recombination frequency zero.
\end{fact}

\subsection{Suppressor Mutations}

\begin{defn}
	A \vocab{suppressor mutation} is one that occurs along with a mutation that causes the mutant to act similarly to the wild-type, i.e. the mutational phenotype is no longer visible. Examples include:
	\begin{itemize}
		\item A \vocab{revertant} mutation is an intragenic mutation in which the wild-type gene is restored (for example TTT (Phe) \to TTA (Leu) \to TTC (Phe)).
		\item A \vocab{informational suppressor} is an extragenic suppressor that is allele-specific and gene non-specific.
		\item A \vocab{bypass suppressor} is an extragenic suppressor that is gene-specific and allele non-specific. 
	\end{itemize}
\end{defn}

\begin{exm}
	A \vocab{Serine tRNA nonsense suppressor} is a mutant in the tRNA gene that recognizes the stop codon UAG as a Serine codon instead. That is, the anticodone AUC now adds a Serine instead of stopping. Note that this is \emph{allele-specific}, since it only recognizes the UCG codon. However, it is \emph{gene non-specific}, since it adds Serine to all UAG's. 
\end{exm}

\begin{fact}*
	In the above example, note that we have only restored wild-type function in genes with Serine \to nonsense mutations. It does not restore wild-type function in general, as it may fail to stop many other translations.
\end{fact}

\begin{exm}*
	Suppose we have a mutation in a maltose channel, but then a lactose permease mutates to also let maltose in. This is a bypass suppressor for the maltose channel mutation.
\end{exm}
