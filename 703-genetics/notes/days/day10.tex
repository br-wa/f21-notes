\section*{Lecture 10: Human Mendelian Traits II (Reddien)}
\setcounter{section}{10}
\resetcounter{subsection}
\resetcounter{defn}
\resetcounter{defncontainer}

\begin{exm}*
	Suppose we have a test for some virus (e.g. HIV) with incidence rate $0.1\%$, and a test with $99.5\%$ accuracy. Let $A$ be the event that someone is infected, and let $B$ be the event that they test positive.
	Then, we can compute: \[
		P(B) = P(B|A) P(A) + P(B|\ol A) P (\ol A) = .995 \cdot 0.001 + 0.005 \cdot 0.999 \approx 0.006. 
	\]
	\[
		P(A|B) = \frac{P(B|A) P(A)}{P(B)} = \frac{0.995 \cdot 0.001}{0.006} \approx 16.6.
	\]
\end{exm}

\subsection{Human Mendelian Traits}

Suppose we have a disease, and we would like to identify its cause.
The first step is to check if something is genetic. We can do this via twin studies:

\begin{defn}*
	A \vocab{twin study} is one in which twins are compared; this helps to separate genetic from environmental factors.
\end{defn}

Next we can ask if it is Mendelian. 
To do this, we can analyze pedigrees, but many traits are complex, so in many cases we find that the inheritance is non-Mendelian.
We will return to these cases later.

Now we aim to find the gene.
There are a few techniques that work here, one of which is sequencing. But sequencing is expensive, so we sequence the exome instead.
Exome sequencing is a bit cheaper, which is basically NGS but where we have probes that hybridize to exome fragments and then wash out the non-hybridized fragments, using the remaining (hybridized and thus exonic fragments) to form the library.

This sort of approach is used for \vocab{rare diseases}, in which the disease has not been studied before. In that case, one does \vocab{trios} sequencing, by sequencing the person with the disease and their parents. The identified variants (including \vocab{de novo} mutations that arise in the child but are not found in the parents) are matched against a database of known variants. Those that are not in the database are correspond to candidate genes that may be involved in the disease.

This is done, for example, in the \emph{Rare Genomes Project} at the Broad.

\subsection{Linkage}

Another method to finding the gene is by using a genetic map. Suppose we have an unknown gene with some allele $D$ associated with some trait. 
We would like to locate $D$, by finding a polymorphism $X$ near it. Possible polymorphisms include:

\begin{itemize}
	\item Single-Nucleotide Polymorphisms (SNPs)
	\item Simple Sequence Repeats (SSRs)
\end{itemize}

SSRs tend to vary a lot throughout the population, because they can appear twice, thrice, four times, etc.

\begin{defn}
	We say a variant is \vocab{highly polymorphic} if it varies a lot throughout the population.
\end{defn}

Note that SNPs are not highly polymorphic, since there are only 4 possible variants.
So, to do a linkage study, we will instead use SSRs.
We take SSRs throughout each chromosome, spaced 10 cM apart.

The way we identify SSR genotypes is by amplifying a given SSR (using e.g. PCR), and then using electrophoresis. The locations of the bands thus reveal the number of repeats.

\subsection{LOD Score}

\begin{defn}
	For two alleles \vocab{log of odds [ratio] (LOD)} score is the (base 10) logarithm of the ratio of the odds that they are linked and the odds that they are unlinked.
\end{defn}	

In particular, let $X$ be the event (SSR marker is linked to the gene), so that $\ol X$ is the event that they are unlinked, and $Y$ is the data we are given. 
Then, \[
	P(X|Y) = \frac{P(Y|X) P(X)}{P(Y)}, P(\ol X|Y) = \frac{P(Y|\ol X) P(\ol X)}{P(Y)},
\]so we must have: \[
	\frac{P(X|Y)}{P(\ol X|Y)} = \frac{P(Y|X) P(X)}{P(Y|\ol X)P(\ol X)}.
\]
Then, note that $\frac{P(X)}{P(\ol X)}$ is essentially our prior, and $\frac{P(Y|X)}{P(Y|\ol X)}$ is calculable from the data.
This gives us the LOD score: 

\begin{fact}
	The LOD score is \[
		\log_{10} \frac{P(\textnormal{data} | \textnormal{linkage at distance } \Theta)}{P(\textnormal{data} | \textnormal{unlinked})}.
	\]
\end{fact}
