\section*{Lecture 4: Recombination and Genetic Maps (Gehring)}
\setcounter{section}{4}
\setcounter{subsection}{0}
\setcounter{defn}{0}
\setcounter{defncontainer}{0}

\subsection{Genetic Maps}

Recall from last time that by measuring the amount of recombination between two genes, we can estimate the ``distance" between genes on a chromosome.
Also, recall that last time by comparing frequencies between different sexes, we were able to conclude that the white-eyed gene is X-linked. 

\begin{fact}*
	The following traits are all X-linked:
	\begin{itemize}
		\item Yellow body (y),
		\item White eyes (w),
		\item Rudimentary wings (r), 
		\item Miniature wings (m), and
		\item Vermillion eyes (v)
	\end{itemize}
\end{fact}

\begin{exper}
	[Sturtevant (1913)]
	Suppose we cross a long-winged, vermillion-eyed true-breeding female with a rudimentary-winged, red-eyed male. This is X\tss{r+;v}X\tss{r+;v} ($\times$) X\tss{r;v+}Y.
	The F\textsubscript1 generation has females with long wings and red eyes (X\tss{r+;v}X\tss{r;v+}) and males with long wings and vermillion eyes (X\tss{r+;v}Y).
	Now we cross such a female with such a male, and observe the male offspring. 
	The possible phenotypes and their proportions among the F\textsubscript2 generation are:
	\begin{itemize}
		\item Long wings, red eyes (X\tss{r+;v+}Y) -- 105,
		\item Long wings, vermillion eyes (X\tss{r+;v}Y) -- 316,
		\item Rudimentary wings, red eyes (X\tss{r;v+}Y) -- 33, and
		\item Rudimentary wings, vermillion eyes (X\tss{r;v}Y) -- 4.
	\end{itemize}
\end{exper}

Note that the parental genes are X\tss{r+;v} and X\tss{r;v+}, we see that the recombination frequency is \[
	\frac{105 + 4}{105 + 4 + 316 + 33} = \frac{109}{458} = 0.238.
\]

\begin{defn}
	If the recombination frequency is $f$, we say the genes are $100f$ \vocab{map units} (or \vocab{centiMorgans (cM)}) apart.
\end{defn}	

\begin{fact}*
	In the above experiment, note that the parental gametes are not equally common, which suggests that the rudimentary mutation is likely fatal.
\end{fact}

\begin{exm}*
	[Sturtevant (1913), Continued]
	The distances between (y,w), (w,m), and (m,y) are 1.0, 33.7, and 34.3 centiMorgans, respectively. 
\end{exm}

We thus conclude that the genes m,y are furthest apart, and w is very close to y but between m and y. Note that the recombination frequency metric is not linear.
In general, if we measure all pairs, then we can similarly order any number of genes.

\subsection{Three-Point Cross}

Note that the recombination frequency does not account for the possibility of two or three etc. crossovers. As a result, the recombination frequency (i.e. distance) between two genes is underestimated.

\begin{exper}
	[Example Three-Point Cross]
	Suppose we have three genes E, F, G with dominant phenotypes EFG and recessive phenotypes efg. We start with crossing EEFFGG ($\x$) eeffgg, which produces F\textsubscript1 generation having genotype EeFfGg.
	We now cross with a \vocab{tester line} eeffgg.
	As before, the phenotype of the F\textsubscript1 children uniquely determine the gamete produced by the EeFfGg genotype parent. 
	We thus count the number of each gamete produced:
	\begin{itemize}
		\item efg 1654 (no recombinations),
		\item efG 9 (E-G and F-G recombinations),
		\item eFg 118 (E-F and F-G recombinations),
		\item eFG 241 (E-F and G-E recombinations),
		\item Efg 252 (E-F and G-E recombinations),
		\item EfG 131 (E-F and F-G recombinations),
		\item EFg 13 (E-G and F-G recombinations),
		\item EFG 1779 (no recombinations).
	\end{itemize}
\end{exper}

Now we analyze the data: We get 742 E-F recombinations, 271 F-G recombinations, and 575 G-E recombinations, which implies that the order of the genes is EGF.
Note that there are 3 classes of gametes:
\begin{itemize}
	\item Parental (most common),
	\item Single Crossovers (E-G + E-F or G-F + E-F).
	\item Double Crossovers (E-G + G-F)
\end{itemize}
Thus, when computing the recombination frequency, we actually add 2(13 + 9) to account for the double crossovers, and then 271 + 575 becomes precisely 742 + 2(13 + 9).
