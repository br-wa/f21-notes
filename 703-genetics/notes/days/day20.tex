\section*{Lecture 20: Gene Regulation (Reddien)}
\setcounter{section}{20}
\resetcounter{subsection}
\resetcounter{defn}
\resetcounter{defncontainer}

\begin{que}
	Different cells have the same genome. 
	So how are they different?
\end{que}

The answer is that in different cells, different genes turn on and off.

\begin{defn}
	The process by which genes are turned on and off is called \vocab{gene regulation}.
\end{defn}

\subsection{Regulatory Mutants}

\begin{exper}
	Perform the following screen: Mutagenize a bunch of haploids, and then grow them on glucose. Then transfer to galactose, and observe which colonies do not grow. Identify one that grows on glucose and not galactose.
	So now we get a bunch of mutants that for some reason do not grow on galactose.
\end{exper}

\begin{exper}*
	Gal1 is a protein that is involved in galactose usage.
	Suppose we grow wild-type yeast in a plate with glucose. At some time $t$, we stop adding glucose, and add galactose instead.
	We can observe that Gal1 expression is low until $t$, at which point it starts increasing. 
	Somehow, the galactose turned on Gal1.
\end{exper}

\begin{defn}
	In the above experiment, if adding galactose turns on Gal1, Gal1 is said to be \vocab{inducble}.
	On the other hand, if adding galactose does not turn on Gal1 (i.e. Gal1 is never on), we say that Gal1 \vocab{uninducible}.
	Finally, if Gal1 is always on, we say that it is \vocab{constitutive}.
\end{defn}

\begin{exper}
	From various screens, we get the following three mutants.
	\begin{itemize}
		\item Gal1- is a Gal deletion (therefore uninducible). Its phenotype is recessive.
		\item M4 is uninducible. Its phenotype is also recessive.
		\item M80 is constitutive.
	\end{itemize}
	We now perform a complementation test, and find that M4 (\x) Gal1- produces inducible diploids. Furthermore, M4 and Gal1- are completely unlinked.
	Thus M4 must be some different gene, which we call Gal4.
	We conclude that Gal4 is a positive regulator of Gal1.

	Similarly, we cross M80 (\x) WT to get an inducible diploid, which implies that M80 is recessive.
	Let Gal80 be the gene carrying M80. By another complementation test we conclude that Gal80 is unlinked to Gal1. 
	Since Gal80 and Gal4 are likely both loss of function with different phenotypes, we can assume they are not the same gene.
\end{exper}

\subsection{Regulatory Pathways}

We have two potential scenarios: \[
	\opera{Gal80} \dashv \opera{Gal4}\to \opera{Gal1}, \qquad \opera{Gal4}\dashv \opera{Gal80}\dashv \opera{Gal1}.
\]
We also have a potential situation where Gal4 and Gal80 are independent of each other.

\begin{defn}
	A \vocab{epistasis test} is used on a single biological step with regulation, in which one makes a double mutant of two individual mutants with \emph{opposing} phenotypes.
	The \vocab{epistatic gene} is the one whose mutant phenotype is observed in the double mutant.
\end{defn}

\begin{exper}
	We now perform an epistasis test on haploids M4 and M80. We get a diploid and do tetrads, and we can get the following potential tetrads:
	\begin{itemize}
		\item The parental ditype (2 uninducible, 2 constitutive).
		\item The non-parental ditype (2 inducible, 2 uninducible).
		\item The tetra-type (2 uninducible, 1 inducible, 1 constitutive).
	\end{itemize}
	We thus conclude that, since the non-parental ditype has two ++ and two double mutants, that the double mutants are uninducible. Thus Gal4 is epistatic.
	It follows that we must have the $\opera{Gal80} \dashv \opera{Gal4} \to \opera{Gal1}$.
\end{exper}

\begin{fact}*
	The epistatic gene is always turned on downstream.
\end{fact}
