\section*{Lecture 1: Introduction to Genetics (Reddien)}
\setcounter{section}{1}

\subsection{Overview}

In general, we will think about genetics as an \emph{information science}.
We will cover the following:
\begin{itemize}
	\item Heredity;
	\item Trait inheritance and disease risk;
	\item Applications (especially medicine, agriculture);
	\item Genetic circuits and regulation, and controlling development and features;
	\item Evolution from a genetic perspective; and
	\item Using genetic techniques to understand other aspects of biology.
\end{itemize}
Piazza exists, although mostly as a forum for students to ask questions for each other. 

Exams are 1 hour long and are take-home. There are six problem sets and the lowest scoring one is dropped (this helps e.g. for extenuating circumstances). 

\subsection{Particulate Inheritance}

\begin{exper}*
	If you cross a red and white snapdragon, you get a pink snapdragon. In general, a lot of traits will end up looking ``mixed" under breeding. 
\end{exper}

This makes things hard to define and study, since different traits seem to ``mix" differently. As a result, we first begin with traits that are \emph{discrete}. 

\begin{defn}
	A \vocab{phenotype} is the characteristic of an organism. Usually this will be the observed characteristic (e.g. ``red phenotype"/``white phenotype"). A phenotype is \vocab{wild-type (WT)} if it the variant that is not of interest.
\end{defn}

Suppose we study \emph{Drosophila} for example, and we find that one fly is ``paralyzed" and does not move well. We can try to explain why this is happening: Some explanations are ``environmental," for example they could be injured or infected or old, while another is that it is inherited.

\begin{defn}
	A \vocab{cross} is an experiment in which two organisms are bred to look at the next generation's traits. The first generation produced is the \vocab{F1} generation, while the second one (produced by crossing two F1 specimens) is \vocab{F2}, and so forth. 
\end{defn}

\begin{exper}*
	We cross a paralyzed and WT fly. The F1 has no paralyzed flies. This does not exclude anything, so we then look at F2. If F2 has a paralyzed fly, then we can reasonably rule out injury and old age. 

	Next, we cross two F2 paralyzed flies. It turns out that all of these offspring are paralyzed! This means that these paralyzed flies are \vocab{true breeding} -- that is, they are ``purely" paralyzed. 
\end{exper}

True breeding strains are very important in genetics, because they give us certainty about how these traits are inherited. 

\begin{defn}
	A \vocab{gene} is a discrete unit of inheritance. An \vocab{allele} of a gene is a variant of the gene. An individual's \vocab{genotype} refers to its set of alleles. Usually, the \vocab{wild-type} allele is the one not of interest.
\end{defn}

Genotyping can also focus on specific genes. For example, we can think about the ``paralyzed" gene from above. We suppose it has two alleles: para\textsuperscript{wt} (the wild-type), and para\textsuperscript{ts} (the paralyzed ones, ``temperature-sensitive" since these flies are only paralyzed at certain temperatures). Most organisms are \vocab{diploid}; that is, they have two copies of each gene. So, we can claim that our true breeding fly has genotype para\textsuperscript{ts}/para\textsuperscript{ts}. 

\begin{exper}
	[para\textsuperscript{ts} causes heat-sensitive paralyzation]
	Now we will illustrate the heat-sensitivity in-class by manipulating some flies under a microscope. First we will take some wild-type para\textsuperscript{wt}/para\textsuperscript{wt} flies, and heat them up to 29 degrees C. They continue to fly around, as expected. Next, the para\textsuperscript{ts}/para\textsuperscript{ts} flies move fine at room temperature, but when we shine the heat lamp on them, they all become paralyzed. Once the heat lamp is removed (and the ambient temperature starts to cool), the para\textsuperscript{ts}/para\textsuperscript{ts} flies start to move again.
\end{exper}

So what does para\textsuperscript{ts} do? The \emph{para} gene encodes a sodium channel that enables nervous system function. Thus, defective ones prevent proper nervous system activity, thus causing paralysis.
