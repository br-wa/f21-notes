\section*{Lecture 13: Mutations and Variations (Gehring)}
\setcounter{section}{13}
\resetcounter{subsection}
\resetcounter{defn}
\resetcounter{defncontainer}

We first clarify some language:

\begin{defn}*
	A \vocab{mutation} is a change in the DNA sequence. A \vocab{mutant} is a strain or individual with a mutation.
\end{defn}

\subsection{Single Nucleotide Mutations}

\begin{defn}
	We define a \vocab{nucleotide mutation} to be a change of one base to another. 
\end{defn}

\begin{exm}
	In coding regions, nucleotide mutations include the following:
	\begin{itemize}
		\item \vocab{Nonsense} (codon becomes a stop codon),
		\item \vocab{Silent/synonymous} (CTT\to CTA [Leu \to Leu] for example, does not change the amino acid),
		\item \vocab{Non-synonymous/missense} (change in amino acid, sometimes does not actually affect the protein).
	\end{itemize}
\end{exm}

\begin{exm}
	Outside of coding regions, we can have the following:
	\begin{itemize}
		\item A \vocab{splice site mutations}, which change how splicing happens.
		\item A \vocab{promoter mutation}, which can change regulation. 
	\end{itemize}
\end{exm}

\begin{fact}
	[Sources of Mutations]
	Mutations can be caused by the following:
	\begin{itemize}
		\item DNA Polymerase error.
		\item Reactive oxygen species (oxidated G pairs with A, deaminated C turns into U, while deaminated methyl-C turns into T).
		\item Radiation (X-ray or UV, latter produces pyrimidine dimers)
		\item Tautomeric shifts (some nucleotide changes isomeric form, which leads the wrong opposing nucleotide to be bound).
		\item Alkylation (adding ethyl group to induce G to A changes) or intercalation (molecules intercalate into the helix, so DNA pol skips bases or adds extras),
		\item Hydrolysis
	\end{itemize}
\end{fact}

\subsection{Chromosomal Aberrations}

\begin{defn}
	A \vocab{chromosomal aberration} is an error that causes a large-scale mutation, usually caused by either errors of segregation, dsDNA breaks, or recombination errors.
\end{defn}

\begin{exm}
	\vocab{Aneuploids} are changes in chromosome number. Examples include trisomy 21 (Down syndrome), XXY (Kleinfelter), XYY (XYY syndrome), or X (Turner syndrome).
	For example, consider the case where after meiosis I we have two copies of chromosome 1 (say A and B) and we now have AABB. Then, we get \vocab{non-disjunction}, in which one daughter cell has AABB and the other does not have chromosome 1. Then, two gametes are AB AB and two do not have a copy of chromosome 1. Then, after fertilization, the AB gametes would lead to trisomy, while the ones without chromsome 1 may lead to a monosomy.

	Non-disjunction can also occur during meiosis II, so that one of the daughter cells has AA and the other has BB, but then the gametes are then \{ AA, $\varnothing$, B, B\}.
\end{exm}

\begin{exm}
	\vocab{Rearrangements} include the following:
	\begin{itemize}
		\item \vocab{Deletions}, in which some segment of the genome gets deleted.
		\item \vocab{Duplications} (usually when crossing over occurs at different points in the two chromosomes) will cause a deletion on one strand and a duplication on the other. For example, suppose ABC x D and AB x CD cross over at the x, then we get ABCCD and ABD.
		\item \vocab{Inversions}, in which two things swap places.  
	\end{itemize}
\end{exm}

\begin{exm}
	\vocab{Translocations} are when parts of a chromosome fuse together. For example, in the \vocab{Philadelphia chromosome}, chromosome 9 fuses with part of 22.
	In the Philadelphia chromosome, BCR (on 22) fuses with ABL (on 9) to create an always-active Tyrosine Kinase, which causes Leukemia.
\end{exm}
