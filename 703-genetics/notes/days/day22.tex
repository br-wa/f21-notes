\section*{Lecture 22: Epigenetics (Gehring)}
\setcounter{section}{22}
\resetcounter{subsection}
\resetcounter{defn}
\resetcounter{defncontainer}

\begin{defn}
	A \vocab{epigenetic} change is one that results in a heritable change (mitotic or meiotic) in phenotype without a DNA difference.
\end{defn}

\begin{exper}*
	In wild-type \emph{A. thaliana} there are 6 stamens. 
	In a variant called \emph{superman (sup)} there are 12 stamens. Since \emph{A. thaliana} is self-fertilizing one does a self-cross when screening, 
	and it was found that in a phenotype called \emph{clark kent (clk)} with a few more stamens (but not as much as \emph{superman}).

	One can then perform a complementation test, whcih verifies that sup/sup (\x) clk/clk indeed produces offspring with extra stamens, which suggests that sup and clk are the same gene.
	On the other hand, adding clk plant to a wild-type sup gene produces wild-type flowers.
	Finally, when looking at RNA expression, clk plants has lower expression levels than wild-type, while sup plants have no expression at all.
	At this point, we might have the following hypothesis: The wild-type superman gene down-regulates stamen regulation, and sup is a deletion, while clk is a hypomorph.

	However, a DNA comparison reveals that sup is a deletion, while clk appears the same as wild-type.
	Since sequencing is no use, one then proceeds to look at DNA methylation, which reveals the following: Clk plants have a methylated sup allele, which prevents full expression.
\end{exper}

\subsection{Epigenetic Modifications}

\begin{defn}
	\vocab{DNA Methylation} is when a methyl group is added to DNA, which generally results in diminished or silenced transcription.
\end{defn}

\begin{defn}
	A \vocab{histone} is an octamer consisting of 2 each of H2A, H2B, H3, H4, and contains lots of lysines (K) and arginines (R), which are basic and thus positively charged. DNA (negatively charged) binds to histones to form \vocab{nucleosomes}.
\end{defn}

\begin{fact}
	Histones have tails, which can be modified. 
\end{fact}

\begin{defn}
	Histones can be modified primarily in two ways:
	\begin{itemize}
		\item \vocab{Methylation} (adding a methyl group) can both activate or silence transcription:
			\begin{itemize}
				\item Methylating H3K9 (lysine 9 on the H3 subunit) generally silences transcription, while
				\item Methylating H3K4 generally activates transcription.
			\end{itemize}
		\item \vocab{Acetylation} (adding a negatively charged acetyl group) generally neutralizes the histone, thus creating \emph{looser chromatin} that activates transcription.
	\end{itemize}
\end{defn}

\begin{defn}
	Nucleosomes are organized into \vocab{chromatin}.
	Chromatin can appear in the following forms:
	\begin{itemize}
		\item \vocab{Euchromatin} is less condensed, so it contains genes that are actively transcripted.
		\item \vocab{Heterochromatin} is more condensed, so its DNA is generally silent. Heterochromatin can be \vocab{constitutive} (such as centromeres, telomeres, and transposons), or \vocab{facultative} (such as developmentally regulated genes).
	\end{itemize}
\end{defn}

\begin{fact}*
	The following are generally true:
	\begin{enumerate}
		\item Silencing often forms a positive feedback loop.
		\item Silencing is often ``turned off" during reproduction (i.e. everything is active).
		\item In certain diseases (e.g. cancer), silencing of certain genes is deactivated, while silencing of other genes (e.g. p53) is activated.
		\item Transcription-directed silencing can affect nearby genes.
	\end{enumerate}
\end{fact}	

\subsection{X-inactivation}

\begin{defn}
	\vocab{X-inactivation} is the process (in most mammals) by which in early female embryos, one of the X chromosomes in each cell is inactivated (iid at random, so that cells have a silenced maternal X, while some have a silenced paternal X). 
		It proceeds via the \vocab{X-inactivation center (Xic)}, which contains two non-coding RNA genes called Xist and Tsix, which are the same region but opposites of each other.
		Via some unknown process called \vocab{transient pairing}, one of the copies of the X chromosome only transcribes Xist, and one only transcribes Tsix.
		Xist then interacts via a histone modification complex that methylates histones (H3K27) on its X chromosome, and Xist also coats its own chromosome, which causes DNA methylation.
		Thus the X chromosome that expressed Xist is silenced.
\end{defn}

\begin{exm}*
	In calico cats, fur color is on the X chromosome, so female cats heterozygous (orange/black) for fur color will have patches of orange fur and patches of black fur.
\end{exm}

\subsection{Imprinting}

\begin{defn}
	\vocab{Imprinting} is a property of certain genes in which only one copy (specific based on the sex of the parent) is active. 
	A genes for which only the maternal copy is active is called a \vocab{maternally expressed gene (MEG)}, 
	while one for which only the paternal copy is active is called a \vocab{paternally expressed gene (PEG)}.
\end{defn}

\begin{fact}*
	Many inherited diseases are on genes with imprinting, so inheritance depends on which parent was a carrier.
\end{fact}
