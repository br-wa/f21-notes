\section*{Lecture 12: Pre-Exam Discussion (Reddien)}
\setcounter{section}{12}
\resetcounter{subsection}
\resetcounter{defn}
\resetcounter{defncontainer}

We return to our example from last time:

\begin{exm}*
	Suppose we have 1 recombinant and 3 non-recombinant. Then we hypothesize a recombinant frequency of $\Theta = 0.25$.
	Then, \[
		P\parens*{\text{data} | \text{linkage at }\Theta} = \parens*{\frac{1-\Theta}{2}}^3 \parens*{\frac{\Theta}2}^1,
	\]
	while $P\parens*{\text{data} | \text{unlinked}} = \parens*{\frac 14}^4$.
\end{exm}

\begin{fact}*
	In general, \[
		\text{LOD} = \text{\# NR} \log_{10}(2-2\Theta) + \text{\# R} \log_{10} (2\Theta).
	\]
\end{fact}

Note that by maximation of cross-entropy, LOD is maximized at $\Theta = \frac{\text{\# R}}{\text{\# R} + \text{\# NR}}$.
In that case, 

\begin{fact}*
	The maximal LOD score is \[
		N\log 2 - N h(\Theta),
	\]where $h(x) = -x\log x - (1-x) \log(1-x)$ is the binary entropy function.
\end{fact}

\begin{defn}
	By convention, we use LOD at least 3 as the threshold for determining linkage.
\end{defn}

It is hard to get LOD 3, so usually we need to study multiple families in order to do this.

\subsection{Unknown Phase}

When we have some informative meioses with a parent with unknown phase, we can add it to the LOD score computation, where we assume the parent is equally likely to be in either phase.

\begin{fact}
	If a parent only contributes one informative meiosis and is of unknown phase, then it contributes $0$ to the LOD score. In particular, this means that it can be ignored.
\end{fact}

We will usually not use these unknown-phase parents to estimate our $\Theta$.
