\section*{Lecture 17: Genetic Analysis I (Reddien)}
\setcounter{section}{17}
\resetcounter{subsection}
\resetcounter{defn}
\resetcounter{defncontainer}

\begin{defn}
	A \vocab{model organism} is used in the lab to model some process we are interested in studying. One factor we especially care about when selecting model organisms is \vocab{generation time}, the time required for a given organism to mature and thus the time between generations.
\end{defn}	

\begin{fact}
	We also care about how well we can pack the organisms, since this limits the number of individuals we can study at a time.
	The following are commonly used as model organisms:
	\begin{center}
		\begin{tabular}{c|c|c|c}	
			Organism & Generation Time & Individuals/ft\tss3 & Model \\ \hline
			\emph{S. cerevisiae} (Yeast) & ~2hr & 10\tss{12} & Eukaryotes \\ \hline
			\emph{C. elegans} (Nematodes) & ~3d & $3\cdot 10^5$ & Animals, Multicellular Organisms \\ \hline
			\emph{D. melanogaster} (Fruit fly) & ~14d & $3\cdot 10^4$ & Animals \\ \hline
			\emph{A. thaliana} (Arabidopsis) & ~7-8wk & 48 & Plants \\ \hline
			\emph{D. rerio} (Zebrafish) & ~3mo & 25 & Vertebrates \\ \hline
			\emph{M. musculus} (Mouse) & ~3mo & 1 & Mammals
		\end{tabular}
	\end{center}
\end{fact}
