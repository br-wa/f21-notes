\section*{Lecture 8: Meiosis and Tetrad Analysis (Reddien)}
\setcounter{section}{8}
\setcounter{subsection}{0}
\setcounter{defn}{0}
\setcounter{defncontainer}{0}

\subsection{Mitosis and Meiosis}

\begin{que}
	What happens during mitosis?
\end{que}

Suppose we have an A/a individual.

\begin{enumerate}
	\item The chromosomes are each replicated, so that we have two A chromosomes and two a chromosomes close to each other. 
	\item Then, two \vocab{mitotic spindles} each pull one chromsome out from each ``pile" for the new cells, so that the new cells are A/a and A/a.
\end{enumerate}

\begin{que}
	What happens during meiosis?
\end{que}

Again, take our A/a individual.

\begin{enumerate}
	\item The chromosomes are again replicated, but instead one has \vocab{homolog pairing}, where the each chromosome (two copies connected at their centromeres) meet with the corresponding replicate from the other parent.
	\item Now, there is \vocab{recombination} between any (A,a) pair.
	\item Then, the spindles act, and this creates an AA and aa cell. This completes \vocab{mitosis I}.
	\item In \vocab{mitosis II}, each of the mitosis I cells replicates, generating haploids.
\end{enumerate}

\subsection{Tetrad Analysis}

\begin{exper}
	It is possible to induce \emph{S. cerevisiae} (baker's yeast) to undergo \vocab{sporulation}, which is essentially meiosis. To perform \vocab{tetrad analysis}, one does the following:
	\begin{enumerate}
		\item Start by crossing two haploid yeast, to yield a diploid. This step is optional.
		\item Induce sporulation in the diploid.
		\item This forms a \vocab{tetrad}, our group of four haploids. Now, one separates these cells, and grows them on separate colonies in a petri dish.
		\item Analyzing the four colonies yields information about meiosis.
	\end{enumerate}
\end{exper}

This is really good!
Instead of looking at random outputs of meiosis, we can look at individual meiosis processes themselves, giving us a very specific and focused tool for analysis.

\begin{exm}*
	[Independent Assortment]
	Suppose A and B are unlinked. 
	Start with an AB (\x) ab cross, which forms an Aa/Bb that undergoes meiosis. Then, after meiosis I, we get either (AA/BB and aa/bb) or (AA/bb and aa/BB), which after meiosis II gives two AB gametes and two ab gametes, or two Ab and two aB gametes.
\end{exm}

\begin{defn}*
	The situation of (2AB, 2ab) is called a \vocab{parental ditype (PD)}, and (2aB, 2Ab) is a \vocab{non-parental ditype (NPD)} 
\end{defn}

\begin{exm}*
	[Linked Genes]
	Suppose A and C are linked, i.e. on the same side of the centromere of the same chromosome.
	Start with an AC (\x) ac cross, which forms an Aa/Cc that undergoes meiosis.
	If there are no crossovers, we get PD.
	If there is a single crossover, we get AC, Ac, aC, and ac children. This is called a \vocab{tetratype (TT)}.
	Now, number the chromosomes 1, 2, 3, 4, so that originally, 1, 2 are AC and 3, 4 are ac. Say AC has a crossover, and that there is a double-crossover.
	We have the following scenarios:
	\begin{itemize}
		\item If the next crossover is 14, then we get NPD.
		\item If the next crossover is 23, then we get PD.
		\item If the next crossover is 13, then we get TT, and
		\item If the next crossover is 24, then we get TT.
	\end{itemize}
	Ignoring double-crossovers, we get: \[
		\textnormal{map distance} = 100\cdot \frac{2\cdot \text{SCOs}}{4\cdot \text{total}} = 50\cdot \frac{\cdot{\text{TT}}}{\text{total}}.
	\]
	If we do add double-crossovers, we get: \[
		\textnormal{map distance} = 100\cdot \frac{2\cdot \text{SCOs} + 4\cdot \text{DCOs}}{4\cdot \text{total}}.
	\]
	The number of DCOs is 4 times the number of NPDs, and the number of SCOs is therefore the number of TTs minus twice the number of NPDs. Thus, \[
		\textnormal{map distance} = 100\cdot \frac{2\cdot \text{TTs} + 12\cdot \text{NPDs}}{4 \cdot \text{total}}.
	\]
\end{exm}
