\section*{Lecture 3: Genetic Linkage (Reddien)}
\setcounter{section}{3}
\setcounter{defn}{0}
\setcounter{subsection}{0}
\setcounter{defncontainer}{0}

\subsection{Complex Inheritance Patterns}

We have mostly been discussing particulate inheritance so far. However, there is another phenomena. 

\begin{exm}*
	[Snapdragons] 
	If you cross a white and red snapdragon, you get a pink snapdragon. 
	In fact, snapdragon color is a case of \vocab{incomplete dominance}, and there is a mendelian gene.
\end{exm}

\begin{exm}*
	[Continuous traits]
	Height is \vocab{polygenic}, i.e. heights are governed by many genes. Continuous traits can also be mostly impacted by \emph{environmental factors} (the effects of which can be studied using e.g. \vocab{twin studies}).
\end{exm}

\subsection{Violations of Independent Assortment}

\begin{exper}
	[Bateson-Saunders-Punnett (1905)]
	We cross pure-breeding pea plants with the following traits: Purple flowers (P) and long pollen (L), and red flowers (p) and short pollen (\ell).
	Crossing, we get an F\textsubscript1 generation with all purple flowers and long pollen, suggesting that the P and L phenotypes are dominant as PpL\ell\, expresses P and L. We then do an F\textsubscript1 ($\x$) F\textsubscript1 cross. We get the following in the F\textsubscript2 generation:
	\begin{itemize}
		\item Purple, Long (P\_;L\_): 1528 peas,
		\item Purple, Short (P\_;\ell\ell): 106 peas,
		\item Red, Long (pp;L\_): 117 peas,
		\item Red, Short (pp;\ell\ell): 381.
	\end{itemize}
\end{exper}

This is clearly not the $9\colon 3\colon 3\colon 1$ ratio predicted by Independent Assortment.
In fact, a $\chi^2$ test (with (1199, 400, 400, 133) expected) gives $p < 0.01$, thus allowing us to reject Independent Assortment.

Note that the F1 PpL\ell\, flies can make either PL, P\ell, pL, or p\ell\, gametes, respectively. Two of these are \vocab{parental} (PL or p\ell) and two are \vocab{recombinant}. Suppose the recombinant alleles are produced with probability $R$ in total (so each gamete appears with probability $R/2$) and the parental alleles are produced with probability $1-R$ (thus each gamete appears with probability $0.5 - R/2$). $R$ is called the \vocab{recombinant frequency}.

Note now that the frequency of the pp;\ell\ell genotype is $\frac{(1-R)^2}{4}$, so we can use the frequency of the red-short peas to estimate $R \approx 0.155$.
This experiment works, but note that if we want to figure out gamete frequency there are other ways to do this. For example, we can do a \begin{center}
	P\sim L/p\sim \ell\, ($\x$) p\sim \ell / p\sim \ell.
\end{center}
Then, the P\sim L, P\sim \ell, p\sim L, and p\sim \ell\, gametes correspond precisely to the (purple, long), (purple, short), (red, long), (red, short) phenotypes.

\subsection{Sex-Linked Traits}

\begin{exper}
	[Morgan (1910)]
	Cross a white-eyed male (white \male) and a red-eyed female fly (red \female). This produces an F\textsubscript1 generation with all red eyes. Next, the F\textsubscript2 generation has:
	\begin{itemize}
		\item \female: 2459 red, 0 white.
		\item \male: 1011 red, 782 white.
	\end{itemize}
	After this, an F\textsubscript1 female and white-eyed F\textsubscript2 male. The resulting phenotypes are:
	\begin{itemize}
		\item \female: half red, half white.
		\item \male: half red, half white.
	\end{itemize}
\end{exper}

In flies, males are XY and females are XX. A hypothesis here is that there is a \emph{white} gene W on the X chromosome (with white-eyed phenotype recessive), where X\tss+ denotes chromosomes with the wild-type allele and X\tss w denotes chromosomes with the white-eyed allele. Then, 
\begin{itemize}
	\item The initial cross is X\tss wY and X\tss+X\tss+.
	\item The F\textsubscript1 generation has X\tss wX\tss+ females, and X\tss+Y males.
	\item The F\textsubscript2 generation has X\tss+X\tss w and X\tss+X\tss+ females (split half-half), and X\tss wY and X\tss+Y males (also half-half).
	\item The F\textsubscript3 generation has X\tss wX\tss w and X\tss wX\tss+ females (split half-half), and X\tss wY and X\tss+Y males (also half-half).
\end{itemize}

To check this hypothesis, cross a white-eyed female with a red-eyed male (thus X\tss wX\tss w and X\tss+Y). The result is all white-eyed males and all red-eyed females, as expected.
