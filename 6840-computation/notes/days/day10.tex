\section*{Lecture 10: Computation History Method}
\setcounter{section}{10}
\resetcounter{subsection}
\resetcounter{defn}
\resetcounter{defncontainer}

We will later study the PCP (Post-Correspondence Problem) from last time. But first we will study a different type of problem.

\begin{defn}
	A \vocab{linearly bounded automaton} (LBA) is a Turing machine where the tape is only big enough to write the input.
\end{defn}

\begin{fact}*
	If $A_{\opera{LBA}}$ is the language consisting of $\vecn{B,w}$ where $B$ is a LBA that accepts $w$, then $A_{\opera{LBA}}$ is decidable.
\end{fact}

\begin{proof}
	You can run the LBA on the input. If some (state, tape string, head position) is reached twice, then you reject (as this represents looping).
\end{proof}

\begin{defn}
	The (state, tape string, head position) data is called a \vocab{configuration}.
\end{defn}

Thus we may rephrase the proof above as: If $N$ is the total number of configurations of the LBA $B$, just run for $N$ iterations.

\begin{fact}*
	$N\leq \abs{Q} \cdot \ell \cdot \abs{\Gamma}^\ell$, where $\ell$ is the input length.
\end{fact}

\begin{defn}
	For a Turing machine $M$ on input $w$, a \vocab{computation history} is a sequence $C_1, C_2, \ldots, C_k$ that $M$ enters when processing $w$ until it halts.
	Since a configuration (where $t_1$ is the stuff left of the head and $t_2$ is the stuff at and to the right of the head) can be written as $t_1qt_2$, where $q$ is the state, we then use this representation and in general write computation histories as $C_1\# \cdots \# C_k$.
\end{defn}

\begin{fact}*
	If $E_{\opera{LBA}}$ is the language consisting of $\vecn B$ where $B$ is an LBA with nonempty language, then $E_{\opera{LBA}}$ is undecidable.
\end{fact}

\begin{proof}
	We will reduce $A_{\opera{TM}}$ to $E_{\opera{LBA}}$. Suppose $R$ is a Turing machine deciding $E_{\opera{LBA}}$. Now we construct $S$ that decides $A_{\opera{TM}}$. Make $S$ do the following on input $\vecn{M,w}$:
	\begin{enumerate}
		\item Let $B_{M,w}$ be an LBA, where $B$ accepts $x$ if $x$ is an accepting computation history of for $M$ on $w$.
		\item Now, just use $R$ to check if $B$'s language is empty (in which case there is no computation history).
	\end{enumerate}	
\end{proof}

%this proof is so good

\begin{defn}
	Let PCP\tss1 denote the version of PCP where we have to start with $\begin{bmatrix} u_1 \\ v_1 \end{bmatrix}$.
\end{defn}

\begin{fact}*
	PCP\tss1 can be reduced to PCP.
\end{fact}

\begin{thm}
	PCP\tss1 is undecidable.
\end{thm}

\begin{proof}
	We will reduce $A_{\opera{TM}}$ to PCP\tss1. Suppose $R$ is a Turing machine deciding PCP. We will create $S$ deciding $A_{\opera{TM}}$ that constructs the appropriate PCP problem.
	$S$ does the following on input $\vecn{M,w}$:
	\begin{enumerate}
		\item Construct a PCP $P_{M,w}$ where a PCP match corresponds to an accepting computation history for $M$ on $w$.
		\item Test using $R$. If there is a match, accept. Else, reject.
	\end{enumerate}
	So how do we construct the PCP?
	We do the following: \[
		\begin{bmatrix} u_1 \\ v_1 \end{bmatrix} = \begin{bmatrix} \# \\ \# q_0 w_1 \cdots w_n \# \end{bmatrix},
	\]
	then furthermore if $\delta(q,a) = (r, b, R)$, then add \[
		\begin{bmatrix} 
			qa \\ br
		\end{bmatrix}.
	\]
	To complete the configuration, for each $a\in \Gamma$ we also need to add $\begin{bmatrix} a \\ a \end{bmatrix}$.
	Similarly, if $\delta(q,a) = (r, b, L)$ and $g\in \Gamma$, add \[
		\begin{bmatrix}
			gqa \\ rgb
		\end{bmatrix}.
	\]
	Finally, for each $a\in \Gamma$, add \[
		\begin{bmatrix} q_{acc}a \\ q_{acc} \end{bmatrix}, 
		\begin{bmatrix} aq_{acc} \\ q_{acc} \end{bmatrix},
		\begin{bmatrix} q_{acc}\#\# \\ \# \end{bmatrix}.
	\]
\end{proof}
