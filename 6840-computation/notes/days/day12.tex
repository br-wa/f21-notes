\section*{Lecture 12: Complexity Theory}
\setcounter{section}{12}
\resetcounter{subsection}
\resetcounter{defncontainer}
\resetcounter{defn}

Today we start on \vocab{complexity theory}, where we are interested in the following problem: Suppose $A$ is a decidable language. How many steps are needed to decide $A$?
We can make a few observations:
\begin{itemize}
	\item This depends on the input, and in particular input size.
	\item This also depends on the computational model.
\end{itemize}

\begin{defn}
	We say a Turing machine $M$ \vocab{runs in time} $t(n)$ ($t\colon \NN \to \NN$) if for any input $w$ of length $n$, $M$ halts on $w$ within $t(n)$ steps.
\end{defn}

\begin{defn}*
	For some $t\colon \NN \to \NN$, define $\opera{TIME}(t(n))$ to be the set of languages $B$ such that some $1$-tape Turing machine decides $B$ and runs in time $O(t(n))$.
\end{defn}

\begin{fact}
	With a 2-tape Turing machine, one can decide $A = \set{a^kb^k\colon k\geq 0}$ can be decided in $O(n)$ time. However, on a one-tape Turing machine, this runtime is not achievable.
\end{fact}

So the time complexity class used above is not ideal, since it seems to depend very strongly on the computational model used.

\begin{thm}
	For $t(n) \geq n$, every mutli-tape Turing machine $M$ that runs in $t(n)$ time has an equivaent $1$-tape turing machine $S$ that runs in time $O(t(n)^2)$. 
\end{thm}

\begin{proof}
	Our simulation strategy from before uses $\leq ct(n)$ (where $c$ is some constant dependent on the number of tapes) steps to simulate a step of $M$.
\end{proof}

This motivates the following:

\begin{defn}
	$\opera{P} = \bigcup\limits_{k\geq 0} \opera{TIME}(n^k) = \opera{TIME}(\opera{poly}(n))$ is the set of \vocab{polynomial time algorithms}.
\end{defn}

In particular, $\opera P$ is independent of ``reasonable" choices of computational models.
Now we consider the following problem:

\begin{defn}*
	Let $\opera{PATH}$ be the following problem: You are given a directed graph. Is there a path from $a$ to $b$?
\end{defn}

\begin{fact}*
	[Dijkstra's/BFS/DFS]
	PATH is in P.
\end{fact}

\begin{defn}
	Let $\opera{HAMPATH}$ be the \vocab{Hamiltonian path} problem: You are given a directed graph. Is there a Hamiltonian path (i.e. one that visits each vertex exactly once) from $a$ to $b$?
\end{defn}

\subsection{Nondeterminism}

\begin{defn}
	A nondeterministic Turing machine $M$ \vocab{runs in time} $t(n)$ if all branches of the computation use $\leq t(n)$ steps for all inputs of length $n$.
	Similarly, define $\opera{TNIME}(t(n))$ as the set of languages $B$ such that some nondeterministic Turing machine decides $B$ and runs in $O(t(n))$ time.
	Finally, define \[
		\opera{NP} = \bigcup_k \opera{NTIME} (n^k).
	\]
\end{defn}

\begin{fact}
	Like $\opera P$, $\opera{NP}$ is invariant across reasonable nondeterministic models of computation.
\end{fact}

\begin{fact}*
	$\opera{HAMPATH}\in \opera{NP}$.
\end{fact}

\begin{fact}
	We have the following intuition: $B\in \opera{NP}$ if and only if it is ``easy to verify" that $w\in B$.
\end{fact}

\begin{proof}
	For the if direction, just try all possible verification branches. For only if, note that if something is in NP, we can just provide the branch of nondeterministic computation tree.
\end{proof}
