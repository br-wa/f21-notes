\section*{Lecture 1: Introduction, Finite Automata, Regular Expressions}
\setcounter{section}{1}

Recitations start tomorrow, and we can go to any. They are optional, but we must go if we are having difficulties; how this is defined is unclear, so we must use our best judgement. Recitations will go over examples and will elaborate on course content; they are targetted towards the middle/bottom of the class, so if we are not having trouble with the material we may find recitations boring.

The required background is comfort with proofs and creative thinking. Collaboration on problem sets is encouraged, but we should \emph{try things on our own first} and \emph{do our own writeups} to make sure you get individual practice as well. This class is on \url{psetpartners.mit.edu}, so we can use that to find collaborators. 

\subsection{Theory of Computation}

Contrasting theoretical CS and theoretical physics is instructive. 
In theoretical physics, usually theorists propose theories, and then the experimentalists try to verify or disprove these theories.
Theoretical CS does not have the same driving role; however, it nevertheless plays a crucial role, notably in cryptography. 

More importantly, theory helps us stay close to the fundamental problems in computer science, such as ``what is intelligence/computation?" 
Theory is also interested in studying what is \emph{feasible} (e.g. fast factoring is famously open) and what is even computable at all. 

\begin{exm}*
	[Famous incomputable problems]
	Certain things cannot be solved via algorithm. For example:
	\begin{itemize}
		\item (Decidability) Given a mathematical statement, is it true or false?
		\item (Halting) Given a program, will it stop?
	\end{itemize}
\end{exm}

Notice that to do these things we need a definition of an algorithm, i.e. a \vocab{model of computation/computational model}. 
Thus, the first half of the class will focus on computability theory -- what is possible. The second half will focus on complexibility theory -- what is practical.

\subsection{Finite Automata/Finite State Machines}

\begin{defn}
	[Finite Automata]
	A \vocab{finite automaton} $M = (Q, \Sigma, \delta, q_0, F)$ consists of the following:
	\begin{itemize}
		\item A finite set $Q$ \vocab{States} $Q = \set{q_1, \ldots, q_N}$.
		\item A finite \vocab{alphabet} $\Sigma$,
		\item A \vocab{transition function} $\delta \colon Q\x \Sigma \to Q$, so that for each state $q_i$ and character $c\in \Sigma$, there exists some other $\delta(q_i, c)\in Q$.
		\item A unique \vocab{start state} $q_0 \in Q$ and some collection of \vocab{accept states} $F\subset Q$.
	\end{itemize}
\end{defn}

Now how do we compute with this? Here is the definition for when the alphabet is just $\set{0,1}$. The general case is similar.

\begin{defn}*
	[Computing with Finite Automata]
	Suppose we have a finite automaton $M$ and a binary input string $S$. Then, the automata does the following:
	\begin{enumerate}
		\item Start at the start state.
		\item For each bit in the input, if the bit is zero, follow the $0$ transition. Otherwise, follow the $1$ transition. (Formally, if the current state is $q$ and the bit is $c$, go to $\delta(q,c)$).
		\item At the end, if we are at an accept state, return \emph{\texttt{accept}}. Else, return \emph{\texttt{reject}}.
	\end{enumerate}
\end{defn}

Note that if the string is $s_1 \cdots s_k$, then can write the transition step  more verbosely as: \[
	q = \delta(\delta(\cdots \delta(q_0, s_1),\cdots, s_{k-1}), s_k),
\]
and then we accept if and only if $q\in F$.

\begin{defn}*
	Let $M$ be a finite automaton, or \vocab{machine}. 
	The \vocab{language} $L(M)$ of $M$ is the set of strings that $M$ accepts. 
	We also sometimes say that $M$ \vocab{recognizes} $L(M)$. 
\end{defn}

In general, any set of strings is called a language. We can ask the following:

\begin{que}*
	What sets of strings (i.e. languages) $A$ can be expressed as $L(M)$ for some machine $M$? Equivalently, which languages can be recognized by some finite automaton?
\end{que}

\begin{defn}
	A language $A$ is \vocab{regular} if it is $L(M)$ for some $M$. That is, a language is regular if it is the language of some finite automaton.
\end{defn}

\begin{fact}*
	The study of finite automata was first employed by linguists (e.g. Chomsky). Thus we use words like alphabets and languages.
\end{fact}

\begin{exm}*
	[Regular languages]
	The following languages are regular:
	\begin{itemize}
		\item (In-Class Example) The set of binary strings with two consecutive $1$s is regular.
		\item (Exercise) The set of strings that start and end with the same letter is regular.
		\item (Multiples of 7) The set of binary strings that represent multiples of $7$ is regular.
	\end{itemize}
\end{exm}

\begin{exm}*
	On the other hand, not all languages are regular: The set of binary strings with the same number of zeros and ones is not regular. 
\end{exm}

\begin{defn}
	[Regular Operations]
	Let $A,B$ be languages. We define the following operations:
	\begin{itemize}
		\item (Union) $A\cup B$ is the set of strings $w$ in either $A$ or $B$.
		\item (Concatenization) $A\circ B$ (or $AB$) is the set of strings of the form $xy$ for $x\in A$ and $y\in B$. 
		\item (Star) $A^*$ is the set of strings that are either the \emph{empty string} (i.e. the string of length zero) or $w = x_1\cdots x_k$ for $k\geq 1$ and$x_1, \ldots, x_k\in A$. Alternatively, we may write: \[
				A^* = \set{\text{\emph{empty string}}} \cup A \cup AA \cup AAA \cup \cdots.
			\]
	\end{itemize}
\end{defn}

Note that the empty string need not be in $A$. On the other hand, the empty string must be in $A^*$.

\subsection{Regular Expressions}

\begin{defn}*
	We will use $\eps$ to denote the empty string and $\varnothing$ to denote the empty language.
\end{defn}

\begin{defn}
	An \vocab{atomic expression} is a member of $\Sigma$, $\Sigma$ itself, $\varnothing$, or $\eps$. A \vocab{regular expression} is an expression made with the atomic expressions and the regular operations.
\end{defn}

\begin{fact}*
	We can actually construct all regular expressions using only the members of $\Sigma$ and $\varnothing$, since $\eps = \varnothing^*$ and $\Sigma$ is just the union of the members of $\Sigma$.
\end{fact}

Each regular expression gives rise to a language, as follows:

\begin{exm}
	[Sigma Star]
	$\Sigma^*$ is language consisting of all finite strings with characters in the alphabet $\Sigma$.
\end{exm}

\begin{exm}*
	On the other hand, suppose $1\in \Sigma$. Then, $\Sigma^* \circ 1 \circ 1 \circ \Sigma^*$ is the language of strings starting with any string in $\Sigma^*$, having two consecutive ones, and then ending with some other string in $\Sigma^*$. That is, it is precisely the language of strings consisting of two consecutive $1$s.
\end{exm}

\subsection{Classifying Regular Languages}

The real kicker is the following:

\begin{thm}
	A language is regular if and only if it can be written as some regular expression.
\end{thm}

We will now aim to prove this. We first need the following:

\begin{lem}
	[Closure]
	The set of regular languages is closed under the regular operations. 
	That is, if $A_1, A_2$ are regular languages with the same alphabet $\Sigma$, so are:
	\begin{itemize}
		\item (Union) $A_1\cup A_2$,
		\item (Concatenation) $A_1\circ A_2$, and
		\item (Star) $A_1^*$.
	\end{itemize}
\end{lem}

\begin{proof}
	[Proof of Union:] Let $L(M_1) = A_1$ and let $L(M_2) = A_2$. The idea is that our new automaton will ``run $M_1$ and $M_2$ simultaneously, and then accept if either $M_1$ accepts or $M_2$ accepts.

	Formally, we define a new finite automaton $M$ whose states are \emph{pairs} $(q, r)$ of states, where $q$ is a state of $M_1$ and $r$ is a state of $M_2$.
	Then, if $\delta_1, \delta_2$ are the transition functions of $M_1, M_2$, define \[
		\delta((q,r), c) = (\delta_1(q,c), \delta_2(r,c)).
	\]
	Also, suppose $q_0$ and $r_0$ are the starting states of $M_1, M_2$. Then, the starting state in $M$ will be $(q_0, r_0)$.
	Combined with $\delta$ just transitioning as $\delta_1, \delta_2$ did, the $q$ and $r$ states transition as they did in $M_1, M_2$. 

	Now, suppose our final state is $(q_f, r_f)$. Then, $q_f$ is the resulting state of running $M_1$ on the input, while $r_f$ is the resulting state of running $M_2$. Thus, to accept the union, $M$ just needs to accept if $q_f\in F_1$ (i.e. $M_1$ accepts) or $r_f\in F_2$ (i.e. $M_2$ accepting).

	Our automaton $M$ now accepts if and only if one of $M_1, M_2$ would have accepted, so $M$ recognizes $A_1\cup A_2$, which must be regular.
\end{proof}


