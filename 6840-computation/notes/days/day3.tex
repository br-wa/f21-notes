\section*{Lecture 3: Non-Regularity, Context-Free Grammars, Pushdown Automata}
\setcounter{section}{3}
\setcounter{subsection}{3}
\setcounter{defn}{0}
\setcounter{defncontainer}{0}

\subsection{Non-Regularity}

Let $B$ be the language of all binary strings containing an equal number of 0s and 1s.
Similarly, let $C$ be the language of all binary strings containing an equal number of ``01"s and "10"s.

Intuivitely, $B$ is not regular, since we need to ``count the 0s and 1s." But this logic does not follow for $C$; in fact, $C$ is regular.
So proving non-regularity is a bit nuanced.

\begin{lem}
	[Pumping Lemma]
	If $A$ is a regular language, then there is a number $p$, the \vocab{pumping length}, such that if $S\in A$ and $\abs S \geq p$, then $S = XYZ$, where
	\begin{itemize}
		\item $XY^iZ\in A$ for all $i \geq 0$,
		\item $Y\neq \eps$, and
		\item $|XY|\leq p$.
	\end{itemize}
\end{lem}

\begin{proof}
	Let $A = L(M)$ for some $M$. Let $p = \abs{Q}$ be the number of states of $M$.
	Say $S\in A$ with $\abs S \geq p$, then let $q_0, q_1, \ldots$ be the sequence of states visited.
	By Pigeonhole, one of $q_i = q_j$ with $0 \leq i < j \leq p$.
	It follows that we can take $X = s_1\cdots s_i, Y = s_{i+1}\cdots s_j, Z = s_{j+1}\cdots$.
\end{proof}

\begin{cor}*
	Let $A = \set{0^k1^k \colon k\geq 0}$ is not regular.
\end{cor}

\begin{proof}
	Take $p$ from the pumping lemma, and take $S = 0^{p+1}1^{p+1}$.
\end{proof}

\begin{cor}*
	$B$ (as defined above) is not regular.
\end{cor}

\begin{proof}
	Suppose $B$ is regular.
	Then, $A = B \cap 0^*1^*$ would be regular, contradiction.
\end{proof}

\subsection{Context-Free Grammars and Pushdown Automata}

\begin{defn}
	A \vocab{context-free grammar (CFG)} $G = (V,\Sigma, R, S$) is given by:
	\begin{itemize}
		\item A finite set $V$ of \vocab{variables},
		\item A finite set $\Sigma$ of \vocab{terminals}, 
		\item A finite set $R$ of \vocab{rules}, which are all of the form $v\to s$ for a variable $v$ and string $s$. These rules can be applied by replacing an instance of $v$ with the string $s$ (for example $uvw \to usw$).
		\item A \vocab{start variable} $S$.
	\end{itemize}
	The \vocab{language} of a grammar $G$ is the set of strings that are reachable by starting with the start variable and applying rules until one gets a string with only terminals. 
\end{defn}

\begin{fact}*
	Every regular language is the language of some grammar $G$. The converse is not true.
\end{fact}

\begin{defn}*
	The generation of a string in the language via some rules is called a \vocab{derivation}. If a string in the language has multiple derivations, the grammar is \vocab{ambiguous}.
\end{defn}

\begin{defn}
	A \vocab{(nondeterministic) pushdown automaton (PDA)} $P = (Q,\Sigma, \Gamma, \delta, q_0, F)$ is given by $Q,\Sigma,q_0,F$ defined similarly to a generic NFA, but $\delta$ is different and $\gamma$ is new:
	\begin{itemize}
		\item There is a stack $S$ now.
		\item $\Gamma$ is the \vocab{stack language}. 
		\item $\delta$ is a function \[
				\delta \colon Q\x (\Sigma \cup \set \eps) \x (\Gamma \cup \set \eps) \to \mathcal P(Q\x (\Gamma\cup\set \eps)),
			\]
			so that the machine transitions by reading or popping the stack (or both or neither), and then going to a new state and (optionally) pushing to the stack.
	\end{itemize}
\end{defn}
