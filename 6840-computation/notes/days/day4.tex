\section*{Lecture 4: Equivalence}
\setcounter{section}{4}
\setcounter{subsection}{0}
\setcounter{defn}{0}
\setcounter{defncontainer}{0}

\begin{thm}
	If $A = L(R)$ for some regular expression $R$, then $A$ is regular.
\end{thm}

\begin{proof}
	It suffices to demonstrate the existence of an NFA $N$ such that $A = L(N)$. We first show this for the atomic expressions, by casework:
	\begin{itemize}
		\item If $R = a\in \Sigma$, then $N$ can just be \[
				\text{start} \stackrel{a}{\longrightarrow} \text{accept}.
			\]
		\item If $R = \eps$, then $N$ can just be \[
				\text{start = accept}.
			\]
		\item If $R = \varnothing$, then $N$ can just be \[
				\text{start} \neq \text{accept}.
			\]
	\end{itemize}
	For larger things, we are just done by closure.
\end{proof}

\begin{defn}
	A \vocab{generalized nondeterministic finite automaton (GNFA)} where transitions are labeled by regular expressions instead of by individual letters.
	As with NFAs, it is allowed to go along any path whose regular expression matches what comes next in the string.

	For this class, we will assume the following:
	\begin{itemize}
		\item There is a unique accept state, and it is not the start state.
		\item For any two states $q_1, q_2$ that are not the start or accept state, there is exactly one arrow $q_1\to q_2$, possibly with the empty expression $\varnothing$ on it.
		\item For any non-start state $q$, there is an arrow $q_0\to q$; furthermore, there are no arrows ending in the start state.
		\item For any non-accept state $q$, if $s$ is the accept state, then there is an arrow $q\to s$; furthermore, there are no arrows starting in the accept state.
	\end{itemize}
\end{defn}

\begin{fact}*
	For any GNFA $G$, there is a GNFA $G'$ with our desired properties.
\end{fact}

\begin{lem}
	For any GNFA $G$ in the specified form above, there is an \vocab{equivalent} regular expression $R$ such that $L(G) = L(R)$.
\end{lem}

\begin{proof}
	Let $k$ be the number of states of $G$. We induct on $k$. In the base case is $k = 2$, i.e. it must be \[
		\text{start} \stackrel{R}{\longrightarrow} \text{accept}.
	\]
	Clearly $L(G) = L(R)$.
	Now, for the inductive step, let $q$ be an arbitrary state that is either accept nor reject. Let $R_q$ be the regular expression on the $q$ self loop. Then, if $R_{ab}$ is the regular expression on $a\to b$, replace $R_{ab}$ with \[
		R_{ab} \cup R_{aq} (R_q)^* R_{qb},
	\]
	and then we can kill $q$ while keeping $L(G)$ the same.
\end{proof}

\begin{cor}*
	For any DFA $M$, $L(M) = L(R)$ for some regular expression $M$.
\end{cor}

\begin{proof}
	Any DFA is a GNFA, which has an equivalent GNFA in our desired form.
\end{proof}

Combining, we get:

\begin{thm}
	A language $L$ is regular if and only if it is $L(R)$ for some regular expression $R$.
\end{thm}

\subsection{Pushdown Automata and Context-Free Grammars}

\begin{fact}*
	If $A$ is a context-free language, then $A$ is the language of some pushdown automaton $P$.
\end{fact}

\begin{proof}
	[Sketch:]
	Say $A$ is the langauge of the context-free grammar $G$.
	Basically, the pushdown automaton does the following:
	\begin{enumerate}
		\item Add the start symbol $S$ to the stack.
		\item Add substitutions to things in the stack (assume we can read and add things to the stack for now) according to the rules, until we get to all terminals.
		\item Check if the pattern matches the input.
	\end{enumerate}
	Of course, this assumption is very big. Essentially, what we do instead is:
	\begin{enumerate}
		\item Add the start symbol $S$ to the stack.
		\item As long as the top of the stack is not a terminal letter, replace by following one of the rules. Once it is a terminal, check with the current letter in the string. If they do not match, return false immediately.
		\item Repeat Step 2 until the whole string is read.
	\end{enumerate}
	This is the same algorithm as before, except now we do not read deep into the stack.
\end{proof}

In fact, the converse is true, so:

\begin{thm}
	$A$ is a context-free language if and only if $A = L(P)$ for some pushdown automaton $P$.
\end{thm}
