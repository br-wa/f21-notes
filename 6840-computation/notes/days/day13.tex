\section*{Lecture 13: Polynomial-Time Reducibility}
\setcounter{section}{13}
\resetcounter{subsection}
\resetcounter{defn}
\resetcounter{defncontainer}

\begin{defn}
	A \vocab{boolean formula} $\gan$ is \vocab{satisfiable} if $\gan$ evaluates to TRUE for some \vocab{assignment} of variables.
	Let SAT be the \vocab{booelean satisfiability problem}, which is the language of $\vecn\gan$ such that $\gan$ is satisfiable.
\end{defn}

\begin{fact}*
	SAT is in NP.
\end{fact}

\begin{thm}
	[Cook-Levin, 1971]
	SAT is in P if and only if P = NP.
\end{thm}

We will not prove this today.

\begin{defn}
	We say $A$ is \vocab{polynomial time reducible to} $B$ (i.e. $A\leq_P B$) if $A$ is mapping reducible by some polynomial time TM.
\end{defn}

Note that we have not defined boolean formulas yet. We do so now:

\begin{defn}
	Define a \vocab{literal} to be a variable $a$ or a negated variable $\ol a$.
	A \vocab{clause} is the OR of some literals.
	A boolean formula is in \vocab{conjunctive normal form (cnf)} if it is written as the AND of some clauses.
\end{defn}

\begin{defn}*
	A boolean formula is a \vocab{3cnf} if it is written as the AND of clauses, which each only have 3 literals. \vocab{3SAT} is the SAT problem but specifically on boolean formulas written as 3cnfs.
\end{defn}

\begin{thm}
	3SAT is polynomial time reducible to CLIQUE, the problem of $\vecn{G,k}$ for graphs $G$ with $k$-cliques.
\end{thm}

\begin{proof}
	The nodes will be each variable in each clause, where we note that a given literal can appear as a node multiple times. We will then connect all pairs, \emph{except} the following:
	\begin{itemize}
		\item If two nodes are in the same clause, do not connect them.
		\item If two nodes represent literals that are complements of each other, do not connect them.
	\end{itemize}
	Now if $\gan$ has $k$ clauses, ask for a $k$-clique. Indeed, satisfying assignments correspond exactly with $k$-cliques.
\end{proof}
