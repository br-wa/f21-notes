\section*{Lecture 16: PSPACE-completeness}
\setcounter{section}{16}
\resetcounter{subsection}
\resetcounter{defn}
\resetcounter{defncontainer}

Last time, we started discussing the following problem:

\begin{defn}*
	Define $\opera{LADDER}_{\opera{DFA}}$ be the language of $\vecn{A,u,v}$ such that there exists $u_0 = u, u_1, \ldots, u_k = v$ such that $u_i \in L(A)$ and $u_i, u_{i+1}$ have the same length, and they differ by at most $1$ character.
\end{defn}

\begin{fact}*
	$\opera{LADDER}_{\opera{DFA}} \in \opera{NPSPACE}$.
\end{fact}

\begin{proof}
	Note that if $m = |u|$, then if such a path $u\to v$ exists, such a path must exist with length $\leq \abs{\Sigma}^m$.
	Thus we can just try all paths starting from $u$ and with length at most $\abs{\Sigma}^m$.
\end{proof}

\begin{thm}
	$\opera{LADDER}_{\opera{DFA}} \in \opera{PSPACE}$.
\end{thm}

\begin{proof}
	We will describe an algorithm that solves the more general problem of identifying languages $\vecn{A,u,v,t}$ such that one can get from $u$ to $v$ with a path of length $\leq t$. On input $\vecn{A,u,v,t}$, 
	\begin{enumerate}
		\item If $t = 1$ accept if $u,v\in A$ have the same length and differ it in at most one subunit.
		\item If $t > 1$, For each $w\in \Sigma^m$, test both $\vecn{A,u,w,(t+1)/2}$, $\vecn{A,w,v,(t+1)/2}$, and accept if both of these accept.
		\item Reject if nothing accepted.
	\end{enumerate}
	The key idea is that the \emph{recursion depth} is $\log_2 t = O(m)$, and each level stores some new $w$, which also only requires space $O(m)$. Thus the total space is $O(m^2)$.
\end{proof}

In fact,

\begin{thm}
	For any $f(n) \geq n$, $\opera{NSPACE}(f(n)) \leq \opera{SPACE}(f(n)^2)$. 
\end{thm}

\begin{proof}
	Let $A\in \opera{NSPACE}(f(n))$ be decided by some non-determinsitic turing machine $N$ in space $f(n)$.
	Then, note that $N$ necessarily finishes in $S(n)$ steps for some $S = O(2^{O(f(n))})$.
	Our deterministic machine $M$ does the following on input $w$:
	\begin{enumerate}
		\item Pick some accepting configuration $C_v$, and let $C_u$ be the starting configuration of $N$ on $w$.
		\item Now, check if one can get from $C_u$ to $C_v$ in $S(n)$ steps (where one uses the algorithm in the above proof).
	\end{enumerate}
	Now we are home, since each state $C_i$ of $N$ only takes $O(f(n))$ space to write, and $\log S(n) = O(f(n))$.
\end{proof}

\begin{cor}*
	$\opera{PSPACE} = \opera{NPSPACE}$.
\end{cor}

\begin{defn}
	Define $\opera{TQBF}$ as the language of \vocab{true quantified boolean formulae}, i.e. true expressions of the form $q_1x_1 \cdot q_nx_n S$, where $S$ is some SAT expression in $x_1,\cdots, x_n$, and $q_i \in \set{\exists, \forall}$.
\end{defn}

\begin{thm}
	$\opera{TQBF}$ is $\opera{PSPACE}$-complete.
\end{thm}

\begin{proof}
	[Sketch:] Use computational history in a manner similar to Cook-Levin. The issue is that the history will be a lot longer (exponentially long, in fact).
	So, we want for each pair $(C,C')$ of configurations and $t$, to be able to write down a (polynomial in $\log t$) sized TQBF $\gan_{C,C',t}$ that encodes ``$M$ goes from $C_i$ to $C_j$ within $t$ steps."
	Indeed, this is just \[
		\gan_{C,C',t} = \exists C_m [\gan_{C,C_m, t/2} \land \gan_{C_m, C, t/2}].
	\]
	Sadly, if you try to just do this, you find that the complexity of this is $f(t) = 2f(t/2)$, which is linear in $t$, which is bad.
	Thus we compactify by \[
		\gan_{C,C',t} = \exists C_m \forall (C_a, C_b)\in \set{(C, C_m), (C_m, C')} \sqrb*{\gan_{C_a, C_b, t/2}},
	\]
	and now this fits in polynomial space.
\end{proof}
